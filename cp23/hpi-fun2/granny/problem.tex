\input{template.tex}

\begin{document}

\makeheader

Madeline has baked a delicious strawberry cake for Granny and now wants to bring it to her as fast as possible.
Both Granny and Madeline live in Celeste National Park, where there are a $n$ bus stations.
These stations are numbered incrementally and connected through $m$ bus lines, each denoting a bus driving from a start to an end station.
Looking on the bus plan, Madeline tries to figure out the fastest way to get to Granny with the cake.
Thanks to the wonderful service at the National Park, there is always a bus available at a station - meaning Madeline won't have to wait at stations.


In order to save even more time, Madeline can instantly dash from one station to another - but only if the station's numbers are adjacent to another.
For example, she can dash from station 5 to either station 4 or station 6.
Since dashing exhausts Madeline very much, she can only dash once.
Madeline currently lives at the first station and Granny lives at the very last station, but since she already visited her in the past, she knows that there is a way for her to get there.
Considering the bus plan, how fast can she reach granny?

\paragraph*{Input}

The first line contains two integers $n$ and $m$ ($2 \leq n \leq 10^6, 1 \leq m \leq 5 \cdot 10^3$): the number of bus stations and connections.
The following $m$ lines each contain 3 integers $s,e$ and $t$ the start and end point of the connection ($1 \leq s,e \leq n$), as well as the time required for the transfer ($1 \leq t \leq 10^5$).

\paragraph*{Output}

Output the length of the shortest path between the first and the $n$-th station.

\begin{samples}
  \sample{sample1}
\end{samples}

\paragraph*{Sample explanation}

Sample 1 explanation: Dash from station 1 to station 2. Then take bus to station 3, which takes 1 time unit, making the whole trip 1 time unit.

\end{document}